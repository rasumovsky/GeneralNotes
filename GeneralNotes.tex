%%%%%%%%%%%%%%%%%%%%%%%%%%%%%%%%%%%%%%%%%
% Arsclassica Article
% LaTeX Template
% Version 1.1 (10/6/14)
%
% This template has been downloaded from:
% http://www.LaTeXTemplates.com
%
% Original author:
% Lorenzo Pantieri (http://www.lorenzopantieri.net) with extensive modifications by:
% Vel (vel@latextemplates.com)
%
% License:
% CC BY-NC-SA 3.0 (http://creativecommons.org/licenses/by-nc-sa/3.0/)
%
% Appropriated by Andrew Hard (15/11/2015)
%
%%%%%%%%%%%%%%%%%%%%%%%%%%%%%%%%%%%%%%%%%

%----------------------------------------------------------------------------------------
%	PACKAGES AND OTHER DOCUMENT CONFIGURATIONS
%----------------------------------------------------------------------------------------

\documentclass[
10pt, % Main document font size
a4paper, % Paper type, use 'letterpaper' for US Letter paper
oneside, % One page layout (no page indentation)
%twoside, % Two page layout (page indentation for binding and different headers)
headinclude,footinclude, % Extra spacing for the header and footer
BCOR5mm, % Binding correction
]{scrartcl}

%%%%%%%%%%%%%%%%%%%%%%%%%%%%%%%%%%%%%%%%%
% Arsclassica Article
% Structure Specification File
%
% This file has been downloaded from:
% http://www.LaTeXTemplates.com
%
% Original author:
% Lorenzo Pantieri (http://www.lorenzopantieri.net) with extensive modifications by:
% Vel (vel@latextemplates.com)
%
% License:
% CC BY-NC-SA 3.0 (http://creativecommons.org/licenses/by-nc-sa/3.0/)
%
%%%%%%%%%%%%%%%%%%%%%%%%%%%%%%%%%%%%%%%%%

%----------------------------------------------------------------------------------------
%	REQUIRED PACKAGES
%----------------------------------------------------------------------------------------

\usepackage[
nochapters, % Turn off chapters since this is an article        
beramono, % Use the Bera Mono font for monospaced text (\texttt)
eulermath,% Use the Euler font for mathematics
pdfspacing, % Makes use of pdftex’ letter spacing capabilities via the microtype package
dottedtoc % Dotted lines leading to the page numbers in the table of contents
]{classicthesis} % The layout is based on the Classic Thesis style

\usepackage{arsclassica} % Modifies the Classic Thesis package

\usepackage[T1]{fontenc} % Use 8-bit encoding that has 256 glyphs

\usepackage[utf8]{inputenc} % Required for including letters with accents

\usepackage{graphicx} % Required for including images
\graphicspath{{Figures/}} % Set the default folder for images

\usepackage{enumitem} % Required for manipulating the whitespace between and within lists

\usepackage{lipsum} % Used for inserting dummy 'Lorem ipsum' text into the template

\usepackage{subfig} % Required for creating figures with multiple parts (subfigures)

\usepackage{amsmath,amssymb,amsthm} % For including math equations, theorems, symbols, etc

\usepackage{varioref} % More descriptive referencing

%----------------------------------------------------------------------------------------
%	THEOREM STYLES
%---------------------------------------------------------------------------------------

\theoremstyle{definition} % Define theorem styles here based on the definition style (used for definitions and examples)
\newtheorem{definition}{Definition}

\theoremstyle{plain} % Define theorem styles here based on the plain style (used for theorems, lemmas, propositions)
\newtheorem{theorem}{Theorem}

\theoremstyle{remark} % Define theorem styles here based on the remark style (used for remarks and notes)

%----------------------------------------------------------------------------------------
%	HYPERLINKS
%---------------------------------------------------------------------------------------

\hypersetup{
%draft, % Uncomment to remove all links (useful for printing in black and white)
colorlinks=true, breaklinks=true, bookmarks=true,bookmarksnumbered,
urlcolor=webbrown, linkcolor=RoyalBlue, citecolor=webgreen, % Link colors
pdftitle={}, % PDF title
pdfauthor={\textcopyright}, % PDF Author
pdfsubject={}, % PDF Subject
pdfkeywords={}, % PDF Keywords
pdfcreator={pdfLaTeX}, % PDF Creator
pdfproducer={LaTeX with hyperref and ClassicThesis} % PDF producer
} % Include the structure.tex file which specified the document structure and layout

\hyphenation{Fortran hy-phen-ation} % Specify custom hyphenation points in words with dashes where you would like hyphenation to occur, or alternatively, don't put any dashes in a word to stop hyphenation altogether

%----------------------------------------------------------------------------------------
%	TITLE AND AUTHOR(S)
%----------------------------------------------------------------------------------------

\title{\normalfont\spacedallcaps{General Notes}} % The article title

\author{\spacedlowsmallcaps{Andrew S. Hard\textsuperscript{1}}} % The article author(s) - author affiliations need to be specified in the AUTHOR AFFILIATIONS block

\date{November 16, 2015} % An optional date to appear under the author(s)

%----------------------------------------------------------------------------------------

\begin{document}

%----------------------------------------------------------------------------------------
%	HEADERS
%----------------------------------------------------------------------------------------

\renewcommand{\sectionmark}[1]{\markright{\spacedlowsmallcaps{#1}}} % The header for all pages (oneside) or for even pages (twoside)
%\renewcommand{\subsectionmark}[1]{\markright{\thesubsection~#1}} % Uncomment when using the twoside option - this modifies the header on odd pages
\lehead{\mbox{\llap{\small\thepage\kern1em\color{halfgray} \vline}\color{halfgray}\hspace{0.5em}\rightmark\hfil}} % The header style

\pagestyle{scrheadings} % Enable the headers specified in this block

%----------------------------------------------------------------------------------------
%	TABLE OF CONTENTS & LISTS OF FIGURES AND TABLES
%----------------------------------------------------------------------------------------

\maketitle % Print the title/author/date block

\setcounter{tocdepth}{2} % Set the depth of the table of contents to show sections and subsections only

\tableofcontents % Print the table of contents

\listoffigures % Print the list of figures

\listoftables % Print the list of tables

%----------------------------------------------------------------------------------------
%	ABSTRACT
%----------------------------------------------------------------------------------------

%\section*{Abstract} % This section will not appear in the table of contents due to the star (\section*)

%\lipsum[1] % Dummy text

%----------------------------------------------------------------------------------------
%	AUTHOR AFFILIATIONS
%----------------------------------------------------------------------------------------

{\let\thefootnote\relax\footnotetext{1 \textit{Department of Physics, University of Wisconsin, Madison, United States of America}}}

%----------------------------------------------------------------------------------------

\newpage % Start the article content on the second page, remove this if you have a longer abstract that goes onto the second page

%----------------------------------------------------------------------------------------
%	INTRODUCTION
%----------------------------------------------------------------------------------------

\section{Introduction}

As a fifth year graduate student, I have come to the realization that significant portions of my skillset and knowledge base are highly specialized. I intend to pursue a career outside of my field of study (experimental high-energy particle physics). In order to enhance my future job prospects, I decided that it would be useful to review basic concepts in computer science, statistics, mathematics, and machine learning. Most of the derivations and explanations will be borrowed from other sources. A list of references is currently being compiled. 

%----------------------------------------------------------------------------------------
% ADD SECTIONS IN INDIVIDUAL FILES
%----------------------------------------------------------------------------------------
%	Statistics
%----------------------------------------------------------------------------------------
\section{Basic Statistics}

%------------------------------------------------
% Deduction and Induction
\subsection{Deduction and Induction}

\textbf{Deduction}: if $A \rightarrow B$ and $A$ is true, then $B$ is true. \\
\textbf{Induction}: if $A \rightarrow B$ and $B$ is true, then $A$ is more plausible. \\

%------------------------------------------------
% Conditional Probability and Bayes Theorem
\subsection{Conditional Probability and Bayes' Theorem}

The \textbf{conditional probability} of an event $A$ assuming that $B$ has occured, denoted $P(A|B)$, is given by:

\begin{equation}
P(A|B)=\frac{P(A \cap B)}{P(B)}.
\end{equation}

Via multiplication:

\begin{equation}
P(A \cap B) = P(A|B)P(B).
\label{equation:pAUB}
\end{equation}

This makes sense. The probability of $A$ \textit{and} $B$ is the probability of $A$ \textit{given} $B$ times the probability of $B$. Or, just as reasonably, the probability of $A$ \textit{and} $B$ is the probability of $B$ \textit{given} $A$ times the probability of $A$. As an aside, the equation above can be generalized:

\begin{equation}
P(A \cap B \cap C) = P(A|B \cap C) P(B \cap C) = P(A|B \cap C) P(B|C) P(C).
\end{equation}

Back to the derivation, since $P(A \cap B) = P(B \cap A)$, we can use Equation \ref{equation:pAUB}:

\begin{equation}
P(A|B)P(B) = P(B|A)P(A).
\end{equation}

And re-arranging gives \textbf{Bayes' Theorem}:
\begin{equation}
P(A|B) = \frac{P(A)P(B|A)}{P(B)} = \frac{P(A)}{P(B)} P(B|A).
\label{equation:simpleBayes}
\end{equation}

% DEFINE prior prob, conditional prob, posterior prob. 
In this equation, $P(A)$ is the \textbf{prior probability}, the initial degree of belief in $A$, or the probability of $A$ before $B$ is observed. It is essentially the probability that hypothesis $A$ is true before evidence is collected. The \textbf{posterior probability} $P(A|B)$ is the degree of belief after taking $B$ into account. 

In a nutshell, Bayes' theorem is a method for calculating the validity of beliefs based on available evidence. "Initial belief plus new evidence equals new and improved belief."

Bayes' theorem describes the probability of an event, based on conditions that might be relevant to the event. With the Bayesian probability interpretation the theorem expresses how a subjective degree of belief should rationally change to account for evidence (\textbf{Bayesian inference}). Note: in the \textbf{Frequentist Interpretation}, probability measures a proportion of outcomes, not probability of belief. 

Bayes' theorem can also be interpreted in the following manner: $A$ is some hypothesis, and $B$ is the evidence for that hypothesis. 

The equation makes intuitive sense, particularly in the last form in which it is presented. If $P(B) \gg P(A)$,  the probability $P(A|B)$ will be small because B \textit{usually} occurs without $A$ occurring. Even if $P(B|A)$ were 100\%, the posterior probability would still be small. On the other hand, if $P(A) \gg P(B)$, $P(A|B)$ is high, even if the conditional probability $P(B|A)$ is smaller. 

Nate Silver has a discussion of Bayesian statistics in "The Signal And The Noise", p245. Let $x$ be the initial estimate of the hypothesis likelihood (\textbf{prior probability}). In the book, $x$ is the initial estimate that your spouse is cheating. $y$ is a conditional probability - the probability of an observation being made given that the hypothesis is true. So $y$ in his case was the probability of underwear appearaing conditional on the spouse cheating. $z$ is the probability of an observation being made given that the hypothesis is false. So $z$ was the probability of underwear appearing if the spouse was not cheating. The \textbf{posterior probability} is the revised estimate of the hypothesis likelihood. In Silver's case, the likelihood that the spouse is cheating on you, given that underwear was found. The posterior probability is given by the expression:

\begin{equation}
x' = \frac{xy}{xy+z(1-x)}.
\end{equation}

In Silver's formulation, the denominator is equivalent to the total probability of an observation being made (total probability of underwear being found). So it is the probability of the hypothesis being true times the associated conditional probability of the observation of underwear plus the probability of the hypothesis being false times the associated conditional probability of the observation.

Note that this method can be applied recursively by substituting $x'$ for $x$ in the expression. We can also come up with a recurrent form of Bayes theorem as formulated in Equation \ref{equation:simpleBayes} by letting $P(A_{n+1}) = P(A_{n}|B)$.

Yet another formulation, which is explained by the logic in the previous paragraph, is given by: 

\begin{equation}
P(A|X) = \frac{P(X|A) P(A)}{P(X|A) P(A) + P(X|not A)P(not A)}
\end{equation}

$P(A|X)$ is the probability of hypothesis $A$, given a positive observation $X$. $P(X|A)$ is the probability of the observation of $X$ under hypothesis $A$. $P(A)$ is the probability of the hypothesis.

The derivation below is from mathworld's article on Bayes' Theorem. In this example, let $A$ and $S$ be sets. Furthermore, let 

\begin{equation}
S \equiv \bigcup_{i=1}^{N} A_{i},
\end{equation}

so that $A_{i}$ is an event in $S$, and $A_{i} \cap A_{j} = \emptyset$ $\forall$ $i \not= j$. Then:

\begin{equation}
A = A \bigcap S = A \bigcap ( \bigcup_{i=1}^{N} A_{i} ) = \bigcup_{i=1}^{N} (A \cap A_{i})
\end{equation}

Then we can compare the probabilities using the Law of Total Probability:

\begin{equation}
P(A) = P(\bigcup_{i=1}^{N} (A \cap A_{i}))=\sum_{i=1}^{N} P(A \cap A_{i}) = \sum_{i=1}^{N} P(A_{i}) P(A|A_{i})
\end{equation}

Using substitution into Bayes' Theorem (Equation \ref{equation:simpleBayes}), we have a new form:

\begin{equation}
P(A_{j} | A) = \frac{P(A_{j}) P(A | A_{j})}{\sum_{i=1}^{N} P(A_{i}) P(A|A_{i})}.
\end{equation}

What is the meaning of this? The denominator is simply $P(A)$, expressed as the sum of its constituents. So this reduces exactly to Bayes theorem as in Equation \ref{equation:simpleBayes}, with $A \rightarrow A_{j}$ and $B \rightarrow A$.

"In many cases, estimating the prior is just guesswork, allowing subjective factors to creep into your calculations. You might be guessing the probability of something that, unlike cancer, does not even exist, such as strings, multiverses, inflation or God. You might then cite dubious evidence to support your dubious belief. In this way, Bayes' theorem can promote pseudoscience and superstition as well as reason." From http://blogs.scientificamerican.com/cross-check/bayes-s-theorem-what-s-the-big-deal/

%------------------------------------------------
% Correlation and Covariance
\subsection{Correlation and Covariance}

Correlation and covariance are similar. Both concepts describe the degree to which two random variables or sets of random variables tend to deviate from their expected values in similar ways.

Let $X$ and $Y$ be two random variables, with means $\mu_{X}$ and $\mu_{Y}$ and standard deviations $\sigma_{X}$ and $\sigma_{Y}$, respectively. The \textbf{covariance} is defined (using $\langle \rangle$ to denote expectation value):

\begin{equation}
\sigma_{XY} = \langle (X - \langle X \rangle) (Y - \langle Y \rangle) \rangle.
\end{equation}

Similarly, the \textbf{correlation} of the two variables is defined:

\begin{equation}
\rho_{XY} = \frac{\langle (X - \langle X \rangle) (Y - \langle Y \rangle) \rangle}{\sigma_{X} \sigma_{Y}}.
\end{equation}

So correlation is dimensionless, while covariance is the multiple of the units of the two variables. \textbf{Variance} is simply the covariance of a variable with itself ($\sigma_{XX}$ is usually denoted $\sigma_{X}^{2}$, the square of the standard deviation). The correlation of a variable with itself is always 1, except in the degenerate case where the two variances are zero and the correlation does not exist.  

% CORRELATION VS COVARIANCE

%----------------------------------------------------------------------------------------
%	Information Theory
%----------------------------------------------------------------------------------------
\section{Information Theory}

Information theory deals with quantifying information. It was developed by Claude Shannon with the goal of establishing limits on signal processing operations such as data compression, reliable storage, and communcation (cite wiki). 

\subsection{Entropy}

Shannon entroy H - the average number of bits per symbol needed to store or communicate a message. "Bits assumes $\log_{2}$. "It is analogous to intensive physical entropies like molar and specific entropy $S_{0}$ but not extensive physical entropy $S$."

There's also entropy H times the number of symbols in a message. This measure of entropy is in bits. It is analogous to absolute (extensive) physical entropy S. 

Landauer's Principle: A change in H*N is equal to a change in physical entropy when the information system is perfectly efficient. 

What does entropy mean? It is a measure of the uncertainty involved in predicting the value of a random variable. Example: a coin flip with 50 \% probabilities of heads or tails provides less information than a die roll with a $\frac{1}{6}$ probability per side. 

The Shannon entropy H in units of bits per symbol is given by

\begin{equation}
H = - \sum_{i} f_{i} \log_{2} (f_{i}) = - \sum_{i} \log_{2} (\frac{c_{i}}{N}).
\end{equation}

In this equation, $i$ is an index over distinct symbols, $f_{i}$ is the frequency of symbols occurring in the message, and $c_{i}$ is the number of times the $i^{th}$ symbol occurs in the message of length $N$. $H N$ is the entropy of a message. This has units of bits (not bits per symbol). 

Example: a Bernoulli (binary) trial where $X=1$ occurs with some probability. The entropy $H(X)$ of the trial is maximized when the two possible outcomes are equally probable ($P(X=1) = 0.5$).

\begin{tabular}{l | l}
\textbf{Boltzmann H} & \textbf{Shannon H} \\
\hline
physical systems of particles & information systems of symbols \\
entropy per particle & entropy per symbol \\
probability of a microstate & probability of a symbol \\
Extensive physical entropy $S$ & Entropy of file or msssage $HN$ \\

\end{tabular}
%----------------------------------------------------------------------------------------
%	Data Structures
%----------------------------------------------------------------------------------------
\section{Data Structures}

%------------------------------------------------
%Subsection: Lists
\subsection{Lists}

Basic List 

Linked Lists

Circular Lists

%------------------------------------------------
% Subsection: Trees
\subsection{Trees}

Binary trees

Balancing

Red-Black trees

%------------------------------------------------
% Subsection: Queues and Stacks
\subsection{Stacks and Queues}

Stacks

Queue

Priority Queue

Max/Min Heap

%------------------------------------------------
% Subsection: Graphs
\subsection{Graphs}

\textbf{Djikstra's algorithm} is an algorithm for finding the shortest paths between nodes in a graph \cite{DijkstraAlgorithm}.

%------------------------------------------------
% Subsection: Hash Maps
\subsection{Hash Maps}


%----------------------------------------------------------------------------------------
%	Numerical Methods
%----------------------------------------------------------------------------------------
\section{Numerical Methods}

%------------------------------------------------
% Subsection: Minimization
\subsection{Minimization}

%------------------------------------------------
%Subsection: Regression
\subsection{Regression}

%------------------------------------------------
% Subsection: Matrix Algebra
\subsection{Matrix Algebra}


%----------------------------------------------------------------------------------------
%	Machine Learning
%----------------------------------------------------------------------------------------
\section{Machine Learning}

%------------------------------------------------
% Subsection: Introduction
\subsection{Introduction}

This section should contain a general description of data science principles, as well as use cases for different algorithms. Many tools for many problems. 

Basic problems in machine learning are classification (labeling) and regression (function estimation). 

How to do importance ranking of features? Feature importance, global loss function.

%------------------------------------------------
% Subsection: Feature Selection
\subsection{Feature Selection}

Garbage in gives garbage out. 
Interactions between features. Interactions are not the same as correlations.
Filters versus wrappers.

%------------------------------------------------
% Subsection: Regression
\subsection{Regression (Again!)}

%------------------------------------------------
% Subsection: Clustering
\subsection{Clustering}

%------------------------------------------------
% Subsection: Support Vector Machines
\subsection{Support Vector Machines}

%------------------------------------------------
% Subsection: Decision Trees
\subsection{Decision Trees}

%------------------------------------------------
% Subsection: Lasso Method
\subsection{Lasso Method}

%----------------------------------------------------------------------------------------
%	Deep Learning
%----------------------------------------------------------------------------------------
\section{Deep Learning}

%------------------------------------------------
%Subsection: Overview
\subsection{Overview}

Explain the use cases of each type of algorithm. 

%------------------------------------------------
%Subsection: Perceptrons
\subsection{Perceptrons}

%------------------------------------------------
%Subsection: MLP: Multi-Layer Perceptrons
\subsection{MLP: Multi-Layer Perceptrons}

%------------------------------------------------
%Subsection: RBM: Restricted Boltzmann Machines
\subsection{RBM: Restricted Boltzmann Machines}

%------------------------------------------------
%Subsection: CNN: Convolutional Neural Networks
\subsection{CNN: Convolutional Neural Networks}

%------------------------------------------------
%Subsection: RNN: Recurrent Neural Networks
\subsection{RNN: Recurrent Neural Networks}

%------------------------------------------------
%Subsection: LSTM: Long Short Term Memory
\subsection{LSTM: Long Short Term Memory}

%------------------------------------------------
%Subsection: Auto-Encoders and Deep Belief Networks
\subsection{Auto-Encoders and Deep Belief Networks}

%------------------------------------------------
%Subsection: Neural Turing Machines
\subsection{Neural Turing Machines}


%----------------------------------------------------------------------------------------
%	Lasso Method
%----------------------------------------------------------------------------------------
\section{Lasso Method}

%\begin{description}
%\item[Word] Definition
%\item[Concept] Explanation
%\item[Idea] Text
%\end{description}

%\begin{itemize}[noitemsep] % [noitemsep] removes whitespace between the items for a compact look
%\item First item in a list
%\item Second item in a list
%\item Third item in a list
%\end{itemize}

%Reference to Table~\vref{tab:label}. % The \vref command specifies the location of the reference

%------------------------------------------------

%\subsection{Figure Composed of Subfigures}

%Reference the figure composed of multiple subfigures as Figure~\vref{fig:esempio}. Reference one of the subfigures as Figure~\vref{fig:ipsum}. % The \vref command specifies the location of the reference

%\begin{figure}[tb]
%\centering
%\subfloat[A city market.]{\includegraphics[width=.45\columnwidth]{Lorem}} \quad
%\subfloat[Forest landscape.]{\includegraphics[width=.45\columnwidth]{Ipsum}\label{fig:ipsum}} \\
%\subfloat[Mountain landscape.]{\includegraphics[width=.45\columnwidth]{Dolor}} \quad
%\subfloat[A tile decoration.]{\includegraphics[width=.45\columnwidth]{Sit}}
%\caption[A number of pictures.]{A number of pictures with no common theme.} % The text in the square bracket is the caption for the list of figures while the text in the curly brackets is the figure caption
%\label{fig:esempio}
%\end{figure}

%----------------------------------------------------------------------------------------
%	References
%----------------------------------------------------------------------------------------

\renewcommand{\refname}{\spacedlowsmallcaps{References}} % For modifying the bibliography heading

\bibliographystyle{unsrt}

\bibliography{bibliography.bib} % The file containing the bibliography

%----------------------------------------------------------------------------------------

\end{document}